%! Author = deepcodec
%! Date = 2022/12/29

\section{Lambda Expressions}\label{ch07:sec:lambda-expressions}
Use lambda expressions where appropriate. Prefer explicit captures when the lambda will escape the current scope.

\subsection{Definition}
Lambda expressions are a concise way of creating anonymous function objects. They're often useful when passing functions as arguments. For example:
% \vspace{-\baselineskip}
\begin{minted}[mathescape,
linenos,
numbersep=5pt,
autogobble, % 左对齐
breaklines,
frame=lines,
framesep=2mm]{C++}
std::sort(v.begin(), v.end(), [](int x, int y) {
return Weight(x) < Weight(y);
});
\end{minted}
They further allow capturing variables from the enclosing scope either explicitly by name, or implicitly using a default capture. Explicit captures require each variable to be listed, as either a value or reference capture:
% \vspace{-\baselineskip}
\begin{minted}[mathescape,
linenos,
numbersep=5pt,
autogobble, % 左对齐
breaklines,
frame=lines,
framesep=2mm]{C++}
int weight = 3;
int sum = 0;
// Captures `weight` by value and `sum` by reference.
std::for_each(v.begin(), v.end(), [weight, &sum](int x) {
sum += weight * x;
});
\end{minted}
Default captures implicitly capture any variable referenced in the lambda body, including \mintinline[breakanywhere,bgcolor=code_bg_pro]{C++}{this} if any members are used:
% \vspace{-\baselineskip}
\begin{minted}[mathescape,
linenos,
numbersep=5pt,
autogobble, % 左对齐
breaklines,
frame=lines,
framesep=2mm]{C++}
const std::vector<int> lookup_table = ...;
std::vector<int> indices = ...;
// Captures `lookup_table` by reference, sorts `indices` by the value
// of the associated element in `lookup_table`.
std::sort(indices.begin(), indices.end(), [&](int a, int b) {
return lookup_table[a] < lookup_table[b];
});
\end{minted}
A variable capture can also have an explicit initializer, which can be used for capturing move-only variables by value, or for other situations not handled by ordinary reference or value captures:
% \vspace{-\baselineskip}
\begin{minted}[mathescape,
linenos,
numbersep=5pt,
autogobble, % 左对齐
breaklines,
frame=lines,
framesep=2mm]{C++}
std::unique_ptr<Foo> foo = ...;
[foo = std::move(foo)] () {
...
}
\end{minted}
Such captures (often called \enquote{init captures} or \enquote{generalized lambda captures}) need not actually \enquote{capture} anything from the enclosing scope, or even have a name from the enclosing scope; this syntax is a fully general way to define members of a lambda object:
% \vspace{-\baselineskip}
\begin{minted}[mathescape,
linenos,
numbersep=5pt,
autogobble, % 左对齐
breaklines,
frame=lines,
framesep=2mm]{C++}
[foo = std::vector<int>({1, 2, 3})] () {
...
}
\end{minted}
The type of a capture with an initializer is deduced using the same rules as \mintinline[breakanywhere,bgcolor=code_bg_pro]{C++}{auto}.


\subsection{Pros}
\begin{itemize}
\item Lambdas are much more concise than other ways of defining function objects to be passed to STL algorithms, which can be a readability improvement.
\item Appropriate use of default captures can remove redundancy and highlight important exceptions from the default.
\item Lambdas, \mintinline[breakanywhere,bgcolor=code_bg_pro]{C++}{std::function}, and \mintinline[breakanywhere,bgcolor=code_bg_pro]{C++}{std::bind} can be used in combination as a general purpose callback mechanism; they make it easy to write functions that take bound functions as arguments.
\end{itemize}

\subsection{Cons}
\begin{itemize}
\item Variable capture in lambdas can be a source of dangling-pointer bugs, particularly if a lambda escapes the current scope.
\item Default captures by value can be misleading because they do not prevent dangling-pointer bugs. Capturing a pointer by value doesn't cause a deep copy, so it often has the same lifetime issues as capture by reference. This is especially confusing when capturing \mintinline[breakanywhere,bgcolor=code_bg_pro]{C++}{this} by value, since the use of \mintinline[breakanywhere,bgcolor=code_bg_pro]{C++}{this} is often implicit.
\item Captures actually declare new variables (whether or not the captures have initializers), but they look nothing like any other variable declaration syntax in C++. In particular, there's no place for the variable's type, or even an \mintinline[breakanywhere,bgcolor=code_bg_pro]{C++}{auto} placeholder (although init captures can indicate it indirectly, e.g., with a cast). This can make it difficult to even recognize them as declarations.
\item Init captures inherently rely on \hyperref[sec:type-deduction-(including-auto)]{type deduction}, and suffer from many of the same drawbacks as \mintinline[breakanywhere,bgcolor=code_bg_pro]{C++}{auto}, with the additional problem that the syntax doesn't even cue the reader that deduction is taking place.
\item It's possible for use of lambdas to get out of hand; very long nested anonymous functions can make code harder to understand.
\end{itemize}

\subsection{Decision}
\begin{itemize}
\item Use lambda expressions where appropriate, with formatting as described \hyperref[sec:lambda-expressions]{below}.
\item Prefer explicit captures if the lambda may escape the current scope. For example, instead of:
% \vspace{-\baselineskip}
\begin{minted}[mathescape,
bgcolor=code_bg_con,
linenos,
numbersep=5pt,
autogobble, % 左对齐
breaklines,
frame=lines,
framesep=2mm]{C++}
{
  Foo foo;
  ...
  executor->Schedule([&] { Frobnicate(foo); })
  ...
}
// BAD! The fact that the lambda makes use of a reference to `foo` and
// possibly `this` (if `Frobnicate` is a member function) may not be
// apparent on a cursory inspection. If the lambda is invoked after
// the function returns, that would be bad, because both `foo`
// and the enclosing object could have been destroyed.
\end{minted}
prefer to write:
% \vspace{-\baselineskip}
\begin{minted}[mathescape,
linenos,
numbersep=5pt,
autogobble, % 左对齐
breaklines,
frame=lines,
framesep=2mm]{C++}
{
  Foo foo;
  ...
  executor->Schedule([&foo] { Frobnicate(foo); })
  ...
}
// BETTER - The compile will fail if `Frobnicate` is a member
// function, and it's clearer that `foo` is dangerously captured by
// reference.
\end{minted}
\item Use default capture by reference (\mintinline[breakanywhere,bgcolor=code_bg_pro]{C++}{[&]}) only when the lifetime of the lambda is obviously shorter than any potential captures.
\item Use default capture by value (\mintinline[breakanywhere,bgcolor=code_bg_pro]{C++}{[=]}) only as a means of binding a few variables for a short lambda, where the set of captured variables is obvious at a glance, and which does not result in capturing \mintinline[breakanywhere,bgcolor=code_bg_pro]{C++}{this} implicitly. (That means that a lambda that appears in a non-static class member function and refers to non-static class members in its body must capture \mintinline[breakanywhere,bgcolor=code_bg_pro]{C++}{this} explicitly or via \mintinline[breakanywhere,bgcolor=code_bg_pro]{C++}{[&]}.) Prefer not to write long or complex lambdas with default capture by value.
\item Use captures only to actually capture variables from the enclosing scope. Do not use captures with initializers to introduce new names, or to substantially change the meaning of an existing name. Instead, declare a new variable in the conventional way and then capture it, or avoid the lambda shorthand and define a function object explicitly.
\item See the section on \hyperref[sec:type-deduction-(including-auto)]{type deduction} for guidance on specifying the parameter and return types.
\end{itemize}
