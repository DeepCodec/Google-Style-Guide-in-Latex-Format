%! Author = deepcodec
%! Date = 2022/12/29

\section{Integer Types}\label{sec:integer-types}
Of the built-in C++ integer types, the only one used is \mintinline[breakanywhere,bgcolor=code_bg_pro]{C++}{int}. If a program needs a variable of a different size, use a precise-width integer type from \mintinline[breakanywhere,bgcolor=code_bg_pro]{C++}{<cstdint>}, such as \mintinline[breakanywhere,bgcolor=code_bg_pro]{C++}{int16_t}. If your variable represents a value that could ever be greater than or equal to $2^{31}$ (2GiB), use a 64-bit type such as \mintinline[breakanywhere,bgcolor=code_bg_pro]{C++}{int64_t}. Keep in mind that even if your value won't ever be too large for an \mintinline[breakanywhere,bgcolor=code_bg_pro]{C++}{int}, it may be used in intermediate calculations which may require a larger type. When in doubt, choose a larger type.

\subsection{Definition}
C++ does not precisely specify the sizes of integer types like \mintinline[breakanywhere,bgcolor=code_bg_pro]{C++}{int}. Typically people assume that \mintinline[breakanywhere,bgcolor=code_bg_pro]{C++}{short} is 16 bits, \mintinline[breakanywhere,bgcolor=code_bg_pro]{C++}{int} is 32 bits, \mintinline[breakanywhere,bgcolor=code_bg_pro]{C++}{long} is 32 bits and \mintinline[breakanywhere,bgcolor=code_bg_pro]{C++}{long long} is 64 bits.

\subsection{Pros}
Uniformity of declaration.

\subsection{Cons}
The sizes of integral types in C++ can vary based on compiler and architecture.

\subsection{Decision}
The standard library header \mintinline[breakanywhere,bgcolor=code_bg_pro]{C++}{<cstdint>} defines types like \mintinline[breakanywhere,bgcolor=code_bg_pro]{C++}{int16_t}, \mintinline[breakanywhere,bgcolor=code_bg_pro]{C++}{uint32_t}, \mintinline[breakanywhere,bgcolor=code_bg_pro]{C++}{int64_t}, etc. You should always use those in preference to \mintinline[breakanywhere,bgcolor=code_bg_pro]{C++}{short}, \mintinline[breakanywhere,bgcolor=code_bg_pro]{C++}{unsigned long long} and the like, when you need a guarantee on the size of an integer. Of the C integer types, only \mintinline[breakanywhere,bgcolor=code_bg_pro]{C++}{int} should be used. When appropriate, you are welcome to use standard types like \mintinline[breakanywhere,bgcolor=code_bg_pro]{C++}{size_t} and \mintinline[breakanywhere,bgcolor=code_bg_pro]{C++}{ptrdiff_t}.

We use \mintinline[breakanywhere,bgcolor=code_bg_pro]{C++}{int} very often, for integers we know are not going to be too big, e.g., loop counters. Use plain old \mintinline[breakanywhere,bgcolor=code_bg_pro]{C++}{int} for such things. You should assume that an \mintinline[breakanywhere,bgcolor=code_bg_pro]{C++}{int} is at least 32 bits, but don't assume that it has more than 32 bits. If you need a 64-bit integer type, use \mintinline[breakanywhere,bgcolor=code_bg_pro]{C++}{int64_t} or \mintinline[breakanywhere,bgcolor=code_bg_pro]{C++}{uint64_t}.

For integers we know can be \enquote{big}, use \mintinline[breakanywhere,bgcolor=code_bg_pro]{C++}{int64_t}.

You should not use the unsigned integer types such as \mintinline[breakanywhere,bgcolor=code_bg_pro]{C++}{uint32_t}, unless there is a valid reason such as representing a bit pattern rather than a number, or you need defined overflow modulo $2^N$. In particular, do not use unsigned types to say a number will never be negative. Instead, use assertions for this.

If your code is a container that returns a size, be sure to use a type that will accommodate any possible usage of your container. When in doubt, use a larger type rather than a smaller type.

Use care when converting integer types. Integer conversions and promotions can cause undefined behavior, leading to security bugs and other problems.

\subsubsection{On Unsigned Integers}
Unsigned integers are good for representing bitfields and modular arithmetic. Because of historical accident, the C++ standard also uses unsigned integers to represent the size of containers - many members of the standards body believe this to be a mistake, but it is effectively impossible to fix at this point. The fact that unsigned arithmetic doesn't model the behavior of a simple integer, but is instead defined by the standard to model modular arithmetic (wrapping around on overflow/underflow), means that a significant class of bugs cannot be diagnosed by the compiler. In other cases, the defined behavior impedes optimization.

That said, mixing signedness of integer types is responsible for an equally large class of problems. The best advice we can provide:
\begin{itemize}
    \item Try to use iterators and containers rather than pointers and sizes.
    \item Try not to mix signedness.
    \item Try to avoid unsigned types (except for representing bitfields or modular arithmetic).
    \item Do not use an unsigned type merely to assert that a variable is non-negative.
\end{itemize}
