%! Author = deepcodec
%! Date = 2022/12/30

\section{Function Argument Comments}\label{sec:function-argument-comments}
When the meaning of a function argument is nonobvious, consider one of the following remedies:
\begin{itemize}
    \item If the argument is a literal constant, and the same constant is used in multiple function calls in a way that tacitly assumes they're the same, you should use a named constant to make that constraint explicit, and to guarantee that it holds.
    \item Consider changing the function signature to replace a \mintinline[breakanywhere,bgcolor=code_bg_pro]{C++}{bool} argument with an \mintinline[breakanywhere,bgcolor=code_bg_pro]{C++}{enum} argument. This will make the argument values self-describing.
    \item For functions that have several configuration options, consider defining a single class or struct to hold all the options, and pass an instance of that. This approach has several advantages. Options are referenced by name at the call site, which clarifies their meaning. It also reduces function argument count, which makes function calls easier to read and write. As an added benefit, you don't have to change call sites when you add another option.
    \item Replace large or complex nested expressions with named variables.
    \item As a last resort, use comments to clarify argument meanings at the call site.
\end{itemize}
Consider the following example:
% \vspace{-\baselineskip}
\begin{minted}[mathescape,
    bgcolor=code_bg_con,
    linenos,
    numbersep=5pt,
    autogobble, % 左对齐
    breaklines,
    frame=lines,
    framesep=2mm]{C++}
// What are these arguments?
const DecimalNumber product = CalculateProduct(values, 7, false, nullptr);
\end{minted}
versus:
% \vspace{-\baselineskip}
\begin{minted}{C++}
ProductOptions options;
options.set_precision_decimals(7);
options.set_use_cache(ProductOptions::kDontUseCache);
const DecimalNumber product =
    CalculateProduct(values, options, /*completion_callback=*/nullptr);
\end{minted}
