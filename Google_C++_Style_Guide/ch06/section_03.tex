%! Author = deepcodec
%! Date = 2022/12/29

\section{Function Overloading}\label{sec:function-overloading}
Use overloaded functions (including constructors) only if a reader looking at a call site can get a good idea of what is happening without having to first figure out exactly which overload is being called.

\subsection{Definition}
You may write a function that takes a const \mintinline[breakanywhere,bgcolor=code_bg_pro]{C++}{std::string&} and overload it with another that takes \mintinline[breakanywhere,bgcolor=code_bg_pro]{C++}{const char*}. However, in this case consider \mintinline[breakanywhere,bgcolor=code_bg_pro]{C++}{std::string_view} instead.
% \vspace{-\baselineskip}
\begin{minted}{C++}
class MyClass {
 public:
  void Analyze(const std::string &text);
  void Analyze(const char *text, size_t textlen);
};
\end{minted}

\subsection{Pros}
Overloading can make code more intuitive by allowing an identically-named function to take different arguments. It may be necessary for templatized code, and it can be convenient for Visitors.

Overloading based on const or ref qualification may make utility code more usable, more efficient, or both. (See \hyperref[ch:tip-of-the-week-148]{TotW \#148} for more.)

\subsection{Cons}
If a function is overloaded by the argument types alone, a reader may have to understand C++'s complex matching rules in order to tell what's going on. Also many people are confused by the semantics of inheritance if a derived class overrides only some of the variants of a function.

\subsection{Decision}
You may overload a function when there are no semantic differences between variants. These overloads may vary in types, qualifiers, or argument count. However, a reader of such a call must not need to know which member of the overload set is chosen, only that \textbf{something} from the set is being called. If you can document all entries in the overload set with a single comment in the header, that is a good sign that it is a well-designed overload set.
