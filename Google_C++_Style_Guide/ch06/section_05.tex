%! Author = deepcodec
%! Date = 2022/12/29

\section{Trailing Return Type Syntax}\label{sec:trailing-return-type-syntax}
Use trailing return types only where using the ordinary syntax (leading return types) is impractical or much less readable.

\subsection{Definition}
C++ allows two different forms of function declarations. In the older form, the return type appears before the function name. For example:
% \vspace{-\baselineskip}
\begin{minted}{C++}
int foo(int x);
\end{minted}

The newer form uses the \mintinline[breakanywhere,bgcolor=code_bg_pro]{C++}{auto} keyword before the function name and a trailing return type after the argument list. For example, the declaration above could equivalently be written:
% \vspace{-\baselineskip}
\begin{minted}{C++}
auto foo(int x) -> int;
\end{minted}

The trailing return type is in the function's scope. This doesn't make a difference for a simple case like \mintinline[breakanywhere,bgcolor=code_bg_pro]{C++}{int} but it matters for more complicated cases, like types declared in class scope or types written in terms of the function parameters.

\subsection{Pros}
Trailing return types are the only way to explicitly specify the return type of a \hyperref[ch07:sec:lambda-expressions]{lambda expression}. In some cases the compiler is able to deduce a lambda's return type, but not in all cases. Even when the compiler can deduce it automatically, sometimes specifying it explicitly would be clearer for readers.

Sometimes it's easier and more readable to specify a return type after the function's parameter list has already appeared. This is particularly true when the return type depends on template parameters. For example:
% \vspace{-\baselineskip}
\begin{minted}{C++}
template <typename T, typename U>
auto add(T t, U u) -> decltype(t + u);
\end{minted}
versus
% \vspace{-\baselineskip}
\begin{minted}{C++}
template <typename T, typename U>
decltype(declval<T&>() + declval<U&>()) add(T t, U u);
\end{minted}

\subsection{Cons}
Trailing return type syntax is relatively new and it has no analogue in C++-like languages such as C and Java, so some readers may find it unfamiliar.

Existing code bases have an enormous number of function declarations that aren't going to get changed to use the new syntax, so the realistic choices are using the old syntax only or using a mixture of the two. Using a single version is better for uniformity of style.

\subsection{Decision}
In most cases, continue to use the older style of function declaration where the return type goes before the function name. Use the new trailing-return-type form only in cases where it's required (such as lambdas) or where, by putting the type after the function's parameter list, it allows you to write the type in a much more readable way. The latter case should be rare; it's mostly an issue in fairly complicated template code, which is \hyperref[sec:template-metaprogramming]{discouraged in most cases}.
