%! Author = deepcodec
%! Date = 2022/12/29

\section{Default Arguments}\label{sec:default-arguments}
Default arguments are allowed on non-virtual functions when the default is guaranteed to always have the same value. Follow the same restrictions as for \hyperref[sec:function-overloading]{function overloading}, and prefer overloaded functions if the readability gained with default arguments doesn't outweigh the downsides below.

\subsection{Pros}
Often you have a function that uses default values, but occasionally you want to override the defaults. Default parameters allow an easy way to do this without having to define many functions for the rare exceptions. Compared to overloading the function, default arguments have a cleaner syntax, with less boilerplate and a clearer distinction between \enquote*{required} and \enquote*{optional} arguments.

\subsection{Cons}
Defaulted arguments are another way to achieve the semantics of overloaded functions, so all the \hyperref[sec:function-overloading]{reasons not to overload functions} apply.

The defaults for arguments in a virtual function call are determined by the static type of the target object, and there's no guarantee that all overrides of a given function declare the same defaults.

Default parameters are re-evaluated at each call site, which can bloat the generated code. Readers may also expect the default's value to be fixed at the declaration instead of varying at each call.

Function pointers are confusing in the presence of default arguments, since the function signature often doesn't match the call signature. Adding function overloads avoids these problems.

\subsection{Decision}
Default arguments are banned on virtual functions, where they don't work properly, and in cases where the specified default might not evaluate to the same value depending on when it was evaluated. (For example, don't write \mintinline[breakanywhere,bgcolor=code_bg_pro]{C++}{void f(int n = counter++);}.)

In some other cases, default arguments can improve the readability of their function declarations enough to overcome the downsides above, so they are allowed. When in doubt, use overloads.
