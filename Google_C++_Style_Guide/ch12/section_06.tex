%! Author = codeclabs-cn
%! Date = 2022/12/30

\section{Floating-point Literals}\label{sec:floating-point-literals}
Floating-point literals should always have a radix point, with digits on both sides, even if they use exponential notation. Readability is improved if all floating-point literals take this familiar form, as this helps ensure that they are not mistaken for integer literals, and that the \mintinline[breakanywhere,bgcolor=code_bg_pro]{C++}{E}/\mintinline[breakanywhere,bgcolor=code_bg_pro]{C++}{e} of the exponential notation is not mistaken for a hexadecimal digit. It is fine to initialize a floating-point variable with an integer literal (assuming the variable type can exactly represent that integer), but note that a number in exponential notation is never an integer literal.
% \vspace{-\baselineskip}
\begin{minted}[mathescape,
    bgcolor=code_bg_con,
    linenos,
    numbersep=5pt,
    autogobble, % 左对齐
    breaklines,
    frame=lines,
    framesep=2mm]{C++}
float f = 1.f;
long double ld = -.5L;
double d = 1248e6;
\end{minted}
% \vspace{-\baselineskip}
\begin{minted}{C++}
float f = 1.0f;
float f2 = 1;   // Also OK
long double ld = -0.5L;
double d = 1248.0e6;
\end{minted}
