%! Author = deepcodec
%! Date = 2022/12/29

\section{Declaration Order}\label{sec:declaration-order}
Group similar declarations together, placing public parts earlier.

A class definition should usually start with a \mintinline[breakanywhere,bgcolor=code_bg_pro]{C++}{public:} section, followed by \mintinline[breakanywhere,bgcolor=code_bg_pro]{C++}{protected:}, then \mintinline[breakanywhere,bgcolor=code_bg_pro]{C++}{private:}. Omit sections that would be empty.

Within each section, prefer grouping similar kinds of declarations together, and prefer the following order:
\begin{enumerate}
    \item Types and type aliases (\mintinline[breakanywhere,bgcolor=code_bg_pro]{C++}{typedef}, \mintinline[breakanywhere,bgcolor=code_bg_pro]{C++}{using}, \mintinline[breakanywhere,bgcolor=code_bg_pro]{C++}{enum}, nested structs and classes, and \mintinline[breakanywhere,bgcolor=code_bg_pro]{C++}{friend} types)
    \item Static constants
    \item Factory functions
    \item Constructors and assignment operators
    \item Destructor
    \item All other functions (\mintinline[breakanywhere,bgcolor=code_bg_pro]{C++}{static} and non-\mintinline[breakanywhere,bgcolor=code_bg_pro]{C++}{static} member functions, and \mintinline[breakanywhere,bgcolor=code_bg_pro]{C++}{friend} functions)
    \item Data members (\mintinline[breakanywhere,bgcolor=code_bg_pro]{C++}{static} and non-\mintinline[breakanywhere,bgcolor=code_bg_pro]{C++}{static})
\end{enumerate}

Do not put large method definitions inline in the class definition. Usually, only trivial or performance-critical, and very short, methods may be defined inline. See \hyperref[sec:inline-functions]{Inline Functions} for more details.

