%! Author = deepcodec
%! Date = 2022/12/29

\section{Doing Work in Constructors}\label{sec:doing-work-in-constructors}
Avoid virtual method calls in constructors, and avoid initialization that can fail if you can't signal an error.

\subsection{Definition}
It is possible to perform arbitrary initialization in the body of the constructor.

\subsection{Pros}
\begin{itemize}
    \item No need to worry about whether the class has been initialized or not.
    \item Objects that are fully initialized by constructor call can be \mintinline[breakanywhere,bgcolor=code_bg_pro]{C++}{const} and may also be easier to use with standard containers or algorithms.
\end{itemize}

\subsection{Cons}
\begin{itemize}
    \item If the work calls virtual functions, these calls will not get dispatched to the subclass implementations. Future modification to your class can quietly introduce this problem even if your class is not currently subclassed, causing much confusion.
    \item There is no easy way for constructors to signal errors, short of crashing the program (not always appropriate) or using exceptions (which are \hyperref[sec:exceptions]{forbidden}).
    \item If the work fails, we now have an object whose initialization code failed, so it may be an unusual state requiring a \mintinline[breakanywhere,bgcolor=code_bg_pro]{C++}{bool IsValid()} state checking mechanism (or similar) which is easy to forget to call.
    \item You cannot take the address of a constructor, so whatever work is done in the constructor cannot easily be handed off to, for example, another thread.
\end{itemize}

\subsection{Decision}
Constructors should never call virtual functions. If appropriate for your code , terminating the program may be an appropriate error handling response. Otherwise, consider a factory function or \mintinline[breakanywhere,bgcolor=code_bg_pro]{C++}{Init()} method as described in \hyperref[ch:tip-of-the-week-42]{TotW \#42}. Avoid \mintinline[breakanywhere,bgcolor=code_bg_pro]{C++}{Init()} methods on objects with no other states that affect which public methods may be called (semi-constructed objects of this form are particularly hard to work with correctly).

