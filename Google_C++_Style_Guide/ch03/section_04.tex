%! Author = deepcodec
%! Date = 2022/12/29

\section{Forward Declarations}\label{sec:forward_declarations}
Avoid using forward declarations where possible. Instead, \hyperref[sec:include-what-you-use]{include the headers you need}.

\subsection{Definition}\label{subsec:definition}
A \enquote{forward declaration} is a declaration of an entity without an associated definition.
% \vspace{-\baselineskip}
\begin{minted}{C++}
// In a C++ source file:
class B;
void FuncInB();
extern int variable_in_b;
ABSL_DECLARE_FLAG(flag_in_b);
\end{minted}

\subsection{Pros}
\begin{itemize}
    \item Forward declarations can save compile time, as \mintinline[breakanywhere,bgcolor=code_bg_pro]{C++}{#include}s force the compiler to open more files and process more input.
    \item Forward declarations can save on unnecessary recompilation. \mintinline[breakanywhere,bgcolor=code_bg_pro]{C++}{#include}s can force your code to be recompiled more often, due to unrelated changes in the header.
\end{itemize}
\subsection{Cons}
\begin{itemize}
    \item Forward declarations can hide a dependency, allowing user code to skip necessary recompilation when headers change.
    \item A forward declaration as opposed to an \mintinline[breakanywhere,bgcolor=code_bg_pro]{C++}{#include} statement makes it difficult for automatic tooling to discover the module defining the symbol.
    \item A forward declaration may be broken by subsequent changes to the library. Forward declarations of functions and templates can prevent the header owners from making otherwise-compatible changes to their APIs, such as widening a parameter type, adding a template parameter with a default value, or migrating to a new namespace.
    \item Forward declaring symbols from namespace \mintinline[breakanywhere,bgcolor=code_bg_pro]{C++}{std::} yields undefined behavior.
    \item It can be difficult to determine whether a forward declaration or a full \mintinline[breakanywhere,bgcolor=code_bg_pro]{C++}{#include} is needed. Replacing an \mintinline[breakanywhere,bgcolor=code_bg_pro]{C++}{#include} with a forward declaration can silently change the meaning of code:
    % \vspace{-\baselineskip}
    \begin{minted}[mathescape,
        linenos,
        numbersep=5pt,
        autogobble, % 左对齐
        breaklines,
        frame=lines,
        framesep=2mm]{C++}
// b.h:
struct B {};
struct D : B {};

// good_user.cc:
#include "b.h"
void f(B*);
void f(void*);
void test(D* x) { f(x); }  // Calls f(B*)
    \end{minted}
    If the \mintinline[breakanywhere,bgcolor=code_bg_pro]{C++}{#include} was replaced with forward decls for \mintinline[breakanywhere,bgcolor=code_bg_pro]{C++}{B} and \mintinline[breakanywhere,bgcolor=code_bg_pro]{C++}{D}, \mintinline[breakanywhere,bgcolor=code_bg_pro]{C++}{test()} would call \mintinline[breakanywhere,bgcolor=code_bg_pro]{C++}{f(void*)}.
    \item Forward declaring multiple symbols from a header can be more verbose than simply \mintinline[breakanywhere,bgcolor=code_bg_pro]{C++}{#include}ing the header.
    \item Structuring code to enable forward declarations (e.g., using pointer members instead of object members) can make the code slower and more complex.
\end{itemize}
\subsection{Decision}
Try to avoid forward declarations of entities defined in another project.