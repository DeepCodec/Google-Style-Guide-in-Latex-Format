%! Author = deepcodec
%! Date = 2022/12/29

\section{Names and Order of Includes}\label{sec:names-and-order-of-includes}
Include headers in the following order: Related header, C system headers, C++ standard library headers, other libraries' headers, your project's headers.

All of a project's header files should be listed as descendants of the project's source directory without use of UNIX directory aliases \mintinline[breakanywhere,bgcolor=code_bg_pro]{C++}{.} (the current directory) or \mintinline[breakanywhere,bgcolor=code_bg_pro]{C++}{..} (the parent directory). For example, \mintinline[breakanywhere,breaklines]{C++}{google-awesome-project/src/base/logging.h} should be included as:
% \vspace{-\baselineskip}
\begin{minted}{C++}
#include "base/logging.h"
\end{minted}
In \mintinline[breakanywhere,bgcolor=code_bg_pro]{C++}{dir/foo.cc} or \mintinline[breakanywhere,bgcolor=code_bg_pro]{C++}{dir/foo_test.cc}, whose main purpose is to implement or test the stuff in \mintinline[breakanywhere,bgcolor=code_bg_pro]{C++}{dir2/foo2.h}, order your includes as follows:

\begin{enumerate}
    \item \mintinline[breakanywhere,bgcolor=code_bg_pro]{C++}{dir2/foo2.h}.
    \item A blank line
    \item C system headers (more precisely: headers in angle brackets with the \mintinline[breakanywhere,bgcolor=code_bg_pro]{C++}{.h} extension), e.g., \mintinline[breakanywhere,bgcolor=code_bg_pro]{C++}{<unistd.h>}, \mintinline[breakanywhere,bgcolor=code_bg_pro]{C++}{<stdlib.h>}.
    \item A blank line
    \item C++ standard library headers (without file extension), e.g., \mintinline[breakanywhere,bgcolor=code_bg_pro]{C++}{<algorithm>}, \mintinline[breakanywhere,bgcolor=code_bg_pro]{C++}{<cstddef>}.
    \item A blank line
    \item Other libraries' \mintinline[breakanywhere,bgcolor=code_bg_pro]{C++}{.h} files.
    \item A blank line
    \item Your project's \mintinline[breakanywhere,bgcolor=code_bg_pro]{C++}{.h} files.
\end{enumerate}
Separate each non-empty group with one blank line.

With the preferred ordering, if the related header \mintinline[breakanywhere,bgcolor=code_bg_pro]{C++}{dir2/foo2.h} omits any necessary includes, the build of \mintinline[breakanywhere,bgcolor=code_bg_pro]{C++}{dir/foo.cc} or \mintinline[breakanywhere,bgcolor=code_bg_pro]{C++}{dir/foo_test.cc} will break. Thus, this rule ensures that build breaks show up first for the people working on these files, not for innocent people in other packages.

\mintinline[breakanywhere,bgcolor=code_bg_pro]{C++}{dir/foo.cc} and \mintinline[breakanywhere,bgcolor=code_bg_pro]{C++}{dir2/foo2.h} are usually in the same directory (e.g., \mintinline[breakanywhere,bgcolor=code_bg_pro]{C++}{base/basictypes_test.cc} and \mintinline[breakanywhere,bgcolor=code_bg_pro]{C++}{base/basictypes.h}), but may sometimes be in different directories too.

Note that the C headers such as \mintinline[breakanywhere,bgcolor=code_bg_pro]{C++}{stddef.h} are essentially interchangeable with their C++ counterparts (\mintinline[breakanywhere,bgcolor=code_bg_pro]{C++}{cstddef}). Either style is acceptable, but prefer consistency with existing code.

Within each section the includes should be ordered alphabetically. Note that older code might not conform to this rule and should be fixed when convenient.

For example, the includes in \mintinline[breakanywhere]{C++}{google-awesome-project/src/foo/internal/fooserver.cc} might look like this:
% \vspace{-\baselineskip}
\begin{minted}{C++}
#include "foo/server/fooserver.h"

#include <sys/types.h>
#include <unistd.h>

#include <string>
#include <vector>

#include "base/basictypes.h"
#include "foo/server/bar.h"
#include "third_party/absl/flags/flag.h"
\end{minted}

\subsection{Exception}
Sometimes, system-specific code needs conditional includes. Such code can put conditional includes after other includes. Of course, keep your system-specific code small and localized. Example:
% \vspace{-\baselineskip}
\begin{minted}{C++}
#include "foo/public/fooserver.h"

#include "base/port.h"  // For LANG_CXX11.

#ifdef LANG_CXX11
#include <initializer_list>
#endif  // LANG_CXX11
\end{minted}
