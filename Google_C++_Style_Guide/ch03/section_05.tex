%! Author = deepcodec
%! Date = 2022/12/29

\section{Inline Functions}\label{sec:inline-functions}
Define functions inline only when they are small, say, 10 lines or fewer.
\subsection{Definition}
You can declare functions in a way that allows the compiler to expand them inline rather than calling them through the usual function call mechanism.
\subsection{Pros}
Inlining a function can generate more efficient object code, as long as the inlined function is small. Feel free to inline accessors and mutators, and other short, performance-critical functions.
\subsection{Cons}
Overuse of inlining can actually make programs slower. Depending on a function's size, inlining it can cause the code size to increase or decrease. Inlining a very small accessor function will usually decrease code size while inlining a very large function can dramatically increase code size. On modern processors smaller code usually runs faster due to better use of the instruction cache.
\subsection{Decision}
A decent rule of thumb is to not inline a function if it is more than 10 lines long. Beware of destructors, which are often longer than they appear because of implicit member- and base-destructor calls!

Another useful rule of thumb: it's typically not cost effective to inline functions with loops or switch statements (unless, in the common case, the loop or switch statement is never executed).

It is important to know that functions are not always inlined even if they are declared as such; for example, virtual and recursive functions are not normally inlined. Usually recursive functions should not be inline. The main reason for making a virtual function inline is to place its definition in the class, either for convenience or to document its behavior, e.g., for accessors and mutators.
